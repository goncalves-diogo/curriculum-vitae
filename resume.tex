\documentclass[]{mcdowellcv}

\usepackage{amsmath}
\usepackage{hyperref}

\name{
    Diogo Gonçalves
}

\contacts{
    diogo.goncaalves@gmail.com
    \linebreak
    github.com/goncalves-diogo
    \linebreak
    goncalves-diogo.github.io
}

\begin{document}

	\makeheader
    
    \begin{cvsection}{Education}
		\begin{cvsubsection}{University of Minho}{}{October 2016 -- April 2022}
			\begin{itemize}
                \item Master’s in Software Engineering (specialized in Computer Graphics and Parallel \& Distributed Computing)
                \item Bachelor's in Software Engineering
			\end{itemize}
		\end{cvsubsection}
	\end{cvsection}
 
	\begin{cvsection}{Employment}
		\begin{cvsubsection}{Bsolus}{}{October 2020 -- November 2021}
			Software Engineer
			\begin{itemize}
                \item Architected a Marketing Automation platform based on \href{https://repositorium.sdum.uminho.pt/handle/1822/80045}{\underline{state-of-the-art}} micro-service patterns
                \item Created CI to static analyze, format, and test the committed code
                \item Developed CD to create a docker image with BitBucket Pipelines and deploy it on GCP
                \item Implemented the system using NestJs, MongoDB, and Docker      
                \item Built layer on top of Jest to reduce workload associated with the development unit and integration tests
			\end{itemize}
		\end{cvsubsection}
	\end{cvsection}

	\begin{cvsection}{Technical Experience}
		 \begin{cvsubsection}{Projects}{}{}
			\begin{itemize}
				\item \href{https://github.com/alves-luis/sdb-zulip_deployment}{\textbf{Zulip}} [Ansible, GCP, Docker]
                    \begin{itemize}
                        \item Deployed Zulip on the GCP with Docker-swarm and ansible
                        \item Configured System monitorization with the ELK stack
                        \item Load tested the deployment with JMeter
                        
                    \end{itemize}
				\item \href{https://github.com/goncalves-diogo/PathTracer}{\textbf{Ray tracer}} [C++, CMake]
                    \begin{itemize}
                        \item Developed a simple Ray tracing based renderer
                        \item Optimized the solution by implementing Russian Roulette and BBVH
                        \item Decreased render time by creating a per-row thread render
                    \end{itemize}	
		        \item \href{https://github.com/maacarvalho/interactive-wind}{\textbf{Interactive Wind}} [GLSL]
                  \begin{itemize}
                      \item Developed a real-time heuristic wind simulation tested on a procedurally generated grass field
                      \item Simulated the wind effect by applying the Navier-Stokes equations
                      \item Implemented obstacles interaction with the voxelization of the objects
                  \end{itemize}
                \item \href{https://github.com/maacarvalho/procedural-terrain-generation}{\textbf{Procedural Terrain Generation}} [GLSL]
                \begin{itemize}
                    \item Developed a real-time procedural terrain generator using simplex noise
                    \item Implemented tessellation detail levels dependent on camera position to increase realism and performance
                    \item Developed a semi-random terrain height system utilizing simplex noise
                \end{itemize}                
                
			\end{itemize}
		\end{cvsubsection}
	\end{cvsection}

	\begin{cvsection}{Skills}
		\begin{cvsubsection}{}{}{}
			\begin{itemize}
				\item \textbf{Programming Languages} C++; C; Python; Typescript; Bash;
				\item \textbf{Frameworks} OpenGL; Jest; NestJS;
				\item \textbf{Developer Tools} Git; Ansible; Agile; Jira; Docker; Postman; Confluence;
			    \item \textbf{Languages} English (C1 level), Portuguese (Native)
			\end{itemize}
		\end{cvsubsection}
	\end{cvsection}
 	\begin{cvsection}{Volunteer Experience}
		\begin{cvsubsection}{CoderDojo}{}{2018-2019}
            Mentor
			\begin{itemize}
				\item CoderDojo is a free, non-profit organization which aims to develop basic programming and algorithmic skills in younger people (ages 7-17).
			\end{itemize}
		\end{cvsubsection}
	\end{cvsection}
\end{document}

