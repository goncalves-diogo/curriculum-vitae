\documentclass[]{mcdowellcv}

\usepackage{amsmath}

\name{
    Diogo Gonçalves
}
\address{
    Braga, Portugal
}
\contacts{
    +351 933777932
    \linebreak
    diogo.goncaalves@gmail.com
    \linebreak
    goncalves-diogo.me
}

\begin{document}

	% Print the header
	\makeheader

	% Print the content
	\begin{cvsection}{Employment}
		\begin{cvsubsection}{Bsolus}{}{October 2020 -- Present}
			Software Engineer
			\begin{itemize}
                \item Designed and developed a Marketing Automation system based on a microservices architecture with horizontal scalability. Developed the CI/CD pipeline integrating with the AWS.
                \item Maintained E-commerce platforms.
                \item Techonologies utilized included: NestJs, TypeScript, Docker, SQL, MongoDB, Terraform, GCP, Bitbucket, Jira, Git, PHP, SQL.
			\end{itemize}
		\end{cvsubsection}
	\end{cvsection}

	\begin{cvsection}{Education}
		\begin{cvsubsection}{University of Minho}{}{2016 -- Present}
			\begin{itemize}
                \item Master’s degree in Informatics Engineering specialized in Computer Graphics and Parallel \& Distributed Computing.
                \item Bachelor's degree in Informatics Engineering, July 2019.
			\end{itemize}
		\end{cvsubsection}
	\end{cvsection}

	\begin{cvsection}{Technical Experience}
		\begin{cvsubsection}{Projects}{}{}
			\begin{itemize}
				\item \textbf{Ray tracer} (2020 - Current). Implemented Ray tracing algorithms in C++ and CUDA, optimized the implement by adding features such as russian roulette, Parallel BBVH, multi-threading.
				\item \textbf{Interactive Wind} (2020). Developed a real time realistic wind simulation in GLSL. Focus of the project was to research known implementations of wind interaction with multiple objects and combine them into a realistic solution to integrate with an engine using shaders.
				\item \textbf{Procedural Terrain Generation} (2019-2020). Developed a real time procedural physically accurate terrain generator. The focus of the project was to research and implement graphical algorithms published on research papares. Then combine them into a procedural terrain generator using shaders.
			\end{itemize}
		\end{cvsubsection}
	\end{cvsection}

	\begin{cvsection}{Technical skills}
		\begin{cvsubsection}{}{}{}
			\begin{itemize}
				\item \textbf{Advanced} C++; C;
				\item \textbf{Intermediate} GLSL; Lua; Bash; Python;
				\item \textbf{Basic} Vimscript; Lua; CUDA;
			\end{itemize}
		\end{cvsubsection}
	\end{cvsection}

	\begin{cvsection}{Languages}
		\begin{cvsubsection}{}{}{}
			\begin{itemize}
				\item Portuguese - Native speaker
				\item English - C1 level
			\end{itemize}
		\end{cvsubsection}
	\end{cvsection}
\end{document}


